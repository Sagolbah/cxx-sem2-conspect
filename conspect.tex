\documentclass[15pt, a4paper]{article}
\usepackage[utf8]{inputenc} % если ваш файл содержит русский текст, нужно указать кодировку
\usepackage[T2A]{fontenc}
\usepackage[russian]{babel} % для того, чтобы писать русский текст
\usepackage{amsmath} % для команды equation*
\usepackage{hyperref} % для вставки ссылок
\usepackage{graphicx}
\usepackage[russian]{babel}
\parindent 0pt
\parskip 0pt
\usepackage{amsmath}
\usepackage{amssymb}
\usepackage{array}
\usepackage[left=2.3cm, right=3.3cm, top=1.7cm, bottom=1.7cm, bindingoffset=0cm]{geometry}
\usepackage{hyperref}
\usepackage{graphicx}
\usepackage{float}
\usepackage{enumitem}
\usepackage{array}
\usepackage{subcaption}
\usepackage{multicol}
\usepackage{fancyhdr} 
\usepackage{extramarks}
\usepackage{todonotes}
\usepackage{color}
\graphicspath{{pictures/}}
\usepackage{lipsum}                     % Dummytext
\usepackage{xargs}                      % Use more than one optional parameter in a new commands
\usepackage{xcolor}

\usepackage{setspace}
\onehalfspacing

\title{C++ year2018, семестр 2}
\author{github.com/Sagolbah/cxx-sem2-conspect}
\date{\today}

\pagestyle{fancy}
\fancyhf{}
\lhead{Нужны контрибьюты!}
\chead{C++ y2018 sem2}
\rhead{\thepage}
\lfoot{github.com/Sagolbah/cxx-sem2-conspect}
\cfoot{}
\rfoot{\today}
\renewcommand\headrulewidth{0.4pt}
\renewcommand\footrulewidth{0.4pt}
\newcommand{\nl}{\newline}
\newcommand{\intba}{\int^b_a}


\setenumerate{topsep=0ex,itemsep=0ex,partopsep=0ex,parsep=0ex}

\begin{document}
	\section{Клуб фанатов Manowar}
	\begin{enumerate}
		\item Каждая секция - отдельная тема у Сорокина (sorokin.github.io/cpp-course).
		\item Для списков используйте \verb|\begin{enumerate}| или \verb|\begin{itemize}| 
		\item Очень помогает latex cheatsheet.
        \item Первые пять билетов пропущены - там асм и прочая дичь. Пока что в приоритете билеты по плюсам.
        \item Может попасться всё до 19 билета включительно.
        \item Писать максимально подробно по каждому пункту - мало ли, что могут спросить.
        \item ГачиБасс228.
    \end{enumerate}
 
    \section{Билет 6}
    \subsection{Структуры}
    С помощью структур можно создавать пользовательские типы. По дефолту все поля структуры публичны,
    в то время как поля классов по дефолту приватны. \nl
    \begin{verbatim}
        //to be continued...
    \end{verbatim}
\end{document}
    